%!TEX root=../document.tex

\section{Einführung}
Diese Übung zeigt die Anwendung von iSCSI und OCFS2.

\subsection{Ziele}
Das Ziel dieser Übung ist einen iSCSI-Initiator zu konfigurieren. Ebenfalls soll ein OCFS2 Cluster konfiguriert werden. 

\subsection{Voraussetzungen}
\begin{itemize}
	\item Grundlagen iSCSI
	\item Grundlagen der Netzwerktechnik
	\item Grundlagen Linux

\end{itemize}
\subsection{Aufgabenstellung}
\subsubsection{Basics}
Installieren und Konfigurieren Sie eine NAS Appliance, welche Block Devices (Festplatten) über iSCSI im Netzwerk freigeben kann (z.B. openfiler, FreeNAS), falls diese nicht bereits für die Übung zur Verfügung gestellt wurde (je nach Gruppe/Zeitplan).

Installieren Sie auf einem Linux-Client einen iSCSI Initiator (bei Debian/Ubuntu im Paket open-iscsi) und konfigurieren Sie diesen entsprechend: zumindest die Authentifizierung muss in der Datei /etc/iscsi/iscsi.conf konfiguriert werden (danach muss open-iscsi neu gestartet werden):

\texttt{node.session.auth.authmethod = CHAP}

\texttt{node.session.auth.username = test}

\texttt{node.session.auth.password = test}

Im Anschluss können Sie den iSCSI Client (Initiator) mit dem iSCSI Server (Target) verbinden. Im ersten Schritt soll der Server (die IP-Adresse muss logischerweise angepasst werden) nach den von ihm angebotenen Targets (spezifischen Block-Devices) befragt werden:

\texttt{iscsiadm -m discovery -t sendtargets -p 10.5.104.33}

Dieser Befehl sollte eine Liste an Targets ausgeben, wie z.B.:

\texttt{iqn.2006-01.com.openfiler:tsn.target01}

\texttt{iqn.2006-01.com.openfiler:tsn.target02}

Danach kann man ein spezifisches Target im Client als "lokale" Festplatte registrieren (bzw. mit \texttt{--logout} wieder entfernen):

\texttt{iscsiadm -m node -T iqn.2006-01.com.openfiler:tsn.target01 -p 15.104.33 -{}-login}

Bei Erfolg sieht man in der Kernel-Console (\texttt{dmesg}) das neue Block-Device wie etwa \texttt{sdb}.

\subsubsection{Selbstständige zu lösende Aufgaben}
\begin{itemize}
	\item Das iSCSI Target soll automatisch beim booten des Clients/Initiators verbunden werden.
	\item Installieren Sie \texttt{ocfs2} auf mindestens 3 Computern (die auf das Block-Devices per iSCSI verbunden sind).
	\item Konfigurieren Sie einen OCFS2 Cluster für diese 3 Computer (Nodes), z.B. über die Datei \texttt{/etc/ocfs2/cluster.conf}.
	\item Formatieren Sie das gemeinsame iSCSI Block Device im OCFS2 Dateisystem (\texttt{mkfs.ocfs2}) - dies muss nur auf einem Computer durchgeführt werden, da es sich ja um ein Cluster-Dateisystem handelt.
	\item Überprüfen Sie, wie sich das Dateisystem bei gemeinsamen Zugriff auf eine Datei verhält.
	\item Dokumentieren Sie Ihre Durchführung und Ihre Ergebnisse.
\end{itemize}
Die Arbeit ist in Gruppen zu 3-4 Teilnehmern möglich. Jeder Teilnehmer muss den iSCSI-Initiator und OCFS2 konfiguriert haben.

Abgabe des Protokolls (1x/Gruppe) als pdf.
\clearpage
